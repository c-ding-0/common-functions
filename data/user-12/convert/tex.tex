% Options for packages loaded elsewhere
\PassOptionsToPackage{unicode}{hyperref}
\PassOptionsToPackage{hyphens}{url}
%
\documentclass[
]{article}
\usepackage{amsmath,amssymb}
\usepackage{lmodern}
\usepackage{ifxetex,ifluatex}
\ifnum 0\ifxetex 1\fi\ifluatex 1\fi=0 % if pdftex
  \usepackage[T1]{fontenc}
  \usepackage[utf8]{inputenc}
  \usepackage{textcomp} % provide euro and other symbols
\else % if luatex or xetex
  \usepackage{unicode-math}
  \defaultfontfeatures{Scale=MatchLowercase}
  \defaultfontfeatures[\rmfamily]{Ligatures=TeX,Scale=1}
\fi
% Use upquote if available, for straight quotes in verbatim environments
\IfFileExists{upquote.sty}{\usepackage{upquote}}{}
\IfFileExists{microtype.sty}{% use microtype if available
  \usepackage[]{microtype}
  \UseMicrotypeSet[protrusion]{basicmath} % disable protrusion for tt fonts
}{}
\makeatletter
\@ifundefined{KOMAClassName}{% if non-KOMA class
  \IfFileExists{parskip.sty}{%
    \usepackage{parskip}
  }{% else
    \setlength{\parindent}{0pt}
    \setlength{\parskip}{6pt plus 2pt minus 1pt}}
}{% if KOMA class
  \KOMAoptions{parskip=half}}
\makeatother
\usepackage{xcolor}
\IfFileExists{xurl.sty}{\usepackage{xurl}}{} % add URL line breaks if available
\IfFileExists{bookmark.sty}{\usepackage{bookmark}}{\usepackage{hyperref}}
\hypersetup{
  hidelinks,
  pdfcreator={LaTeX via pandoc}}
\urlstyle{same} % disable monospaced font for URLs
\usepackage{longtable,booktabs,array}
\usepackage{calc} % for calculating minipage widths
% Correct order of tables after \paragraph or \subparagraph
\usepackage{etoolbox}
\makeatletter
\patchcmd\longtable{\par}{\if@noskipsec\mbox{}\fi\par}{}{}
\makeatother
% Allow footnotes in longtable head/foot
\IfFileExists{footnotehyper.sty}{\usepackage{footnotehyper}}{\usepackage{footnote}}
\makesavenoteenv{longtable}
\usepackage{graphicx}
\makeatletter
\def\maxwidth{\ifdim\Gin@nat@width>\linewidth\linewidth\else\Gin@nat@width\fi}
\def\maxheight{\ifdim\Gin@nat@height>\textheight\textheight\else\Gin@nat@height\fi}
\makeatother
% Scale images if necessary, so that they will not overflow the page
% margins by default, and it is still possible to overwrite the defaults
% using explicit options in \includegraphics[width, height, ...]{}
\setkeys{Gin}{width=\maxwidth,height=\maxheight,keepaspectratio}
% Set default figure placement to htbp
\makeatletter
\def\fps@figure{htbp}
\makeatother
\setlength{\emergencystretch}{3em} % prevent overfull lines
\providecommand{\tightlist}{%
  \setlength{\itemsep}{0pt}\setlength{\parskip}{0pt}}
\setcounter{secnumdepth}{-\maxdimen} % remove section numbering
\ifluatex
  \usepackage{selnolig}  % disable illegal ligatures
\fi

\author{}
\date{}

\begin{document}

{
\setcounter{tocdepth}{3}
\tableofcontents
}
\begin{quote}
There she stood in the doorway\\
I heard the mission bell\\
And I was thinking to myself\\
``This could be Heaven or this could be Hell''\\
\(\tiny{\text{The Eagles - Hotel California}}\)
\end{quote}

Infinity in mathematics is a topic that I often see raising curiosity in
non mathematicians. And why wouldn't it? It seems mysterious and is
often attributed mystical implications.\\
This post will be an attempt to explain a few definitions and properties
of cardinal numbers, which are a type of numbers that reach ``to
infinity and beyond.'' I intend to give meaning to this and explain some
(and by some I mean a very small fraction) of what can and what can't be
done with those concepts.

Now, there are many different things that we call ``infinite'' in
mathematics. For instance, a ``point at infinity'' in geometry has
little to do (to my knowledge at least) with the topic being discussed
here. So I should disclaim right away that this is not an explanation to
the whole concept of infinity in mathematics. This concept is far too
vast to be covered in one blog post. Hopefully, I'll be able to talk
about other occurences of infinity in math in later discussions.

\hypertarget{what-are-cardinal-numbers}{%
\subsubsection{What are cardinal
numbers?}\label{what-are-cardinal-numbers}}

Numbers have two different fundamental uses that common language itself
distinguishes pretty eloquently :\\
There are cardinal numbers. We use them to count things. If I claim that
I own \emph{three} apples, I am refering to the cardinal number 3.\\
And then, there are ordinal numbers. Their purpose is to order thing. If
I claim that I am \emph{third} in line, I am refering to the ordinal
number 3.\\
Of course, when we are manipulating finite numbers, distinguishing
between cardinal and ordinal number is just a matter of precision. The
numbers themselves are the same. But things do not go so well when we
consider infinite numbers.\\
This post will focus on \textbf{cardinal numbers}.

In mathematics a \emph{set} is an object that can contain other objects.
Sets are the basic brick on which lay modern mathematics (although other
foundations are possible and studied, set theory is the most common).
And cardinal numbers in mathematics are a measure of the size of sets.\\
For instance, \(\{1;2;3\}\) is the set that contains the numbers 1, 2
and 3. Its cardinal is 3, because it contains 3 distinct elements. Here
is a visual representation of this set:\\
\includegraphics{https://functors.net/montessiel/wp-content/uploads/sites/11/2021/03/3.png}

Now, sets can contain many things aside from numbers, such as functions,
vectors, other sets\ldots{} In fact, assuming you are doing mathematics
based on set theory (which is by far the most common manner), every
mathematical object is a set. There are a few rules to keep things
coherent, such as the fact that a set may not contain itself.

We are trying to define cardinal numbers. That is, we are trying to give
meaning to the sentence ``the set \(\{cow;potato; truck\}\) has 3
elements.'' In order to do that, we are going to use our set
\(\{1;2;3\}\) as a reference : as set will be said to have 3 elements if
it has \emph{as many elements} as the set \(\{1;2;3\}\). But this
tentative definition raises a new question : what those it mean for two
sets to have ``as many elements'' ?

Of course, we could be tempted to just count. The set
\(\{cow, potato, truck\}\) has 3 elements, just like \(\{1;2;3\}\). But
then we would be going in circles, as we are trying do define what it
means to have 3 elements. And circular definitions are to be avoided at
all cost! Instead, we will say that two sets have as many elements, or
that they have \emph{the same cardinal} if there exists a
\emph{bijection} between them. A bijection will be a one-to-one
correspondance between all objects of the first set, and all objects of
the second set. Allow me to demonstrate:\\
\includegraphics{https://functors.net/montessiel/wp-content/uploads/sites/11/2021/03/bij.png}\\
Here, we have drawn a bijection that associates 1 to cow, 2 to potato
and 3 to truck. Of course, we could have made other choices, and if I
had decided instead to associate 1 to potato and 2 to cow, the result
would have been the same:\\
\textbf{\emph{Each element of the first set is associated with a unique
element of the second set, and likewise every element of the second set
is associated with a unique element of the first set}}\\
We will say that two sets have the \emph{same cardinal}, that is, that
they have as many elements, if it is possible to find a bijection
between them. In particular, we can say that a set has 7 elements if
there is a bijection between said set and \(\{1;2;3;4;5;6;7\}\).\\
We may now proceed to introduce infinite sets.

\hypertarget{to-infinity}{%
\subsubsection{To infinity\ldots{}}\label{to-infinity}}

We explained in the last section what it means for a set to have a
certain numbers of elements:\\
If \(n\) is a whole number, a set \(A\) has \(n\) elements if there
exists a bijection between \(A\) and the set \(\{1;2;...;n\}\).\\
But what of the set \(\mathbb{N}\) of whole numbers? That is,
\(\mathbb{N} = \{1;2;3;...\}\) is the set that contains every positive
number. Certainly, we can not assign it a finite number of elements. By
its very definition, it will go beyond that. \(\mathbb{N}\) is our first
infinite set, and so its number of elements is our fist cardinal number.
It is often denoted \(\aleph_0\) (read ``aleph nought''). This simply
mean that this is the \emph{smallest infinite cardinal number}.

And you may have noticed that what I just wrote implied that there are
more infinite cardinals. Considering I have been bragging about going
``to infinity and beyon,'' this may not be much of a surprised, but I
believe that this whole idea can be hard to grasp at first. Fine, let's
experiment!

In order to understand our bijections, we are going to proceed with the
following experiment:\\
When we compare two sets, one will play the role of the \textbf{hotel}
and the second will be the set of \textbf{visitors}. Then, following our
earlier definitions, we will say that our sets have same size if we can
attribute one room to each visitor in a way that will make the hotel
complete. This thought experiment is commonly refered to as ``Hilbert's
Hotel,'' which explains the title of this post. Here is a photograph I
took during my stay at Hotel Hilbert. Unfortunately, I wasn't able to
fit the whole building in the picture.\\
\includegraphics{https://functors.net/montessiel/wp-content/uploads/sites/11/2021/03/hotel-hilbert-1.png}\\
Now, let's put our thought experiment to use. I use you to take a small
pause and ponder the following question for yourself before looking at
the answer:\\
Do the sets \(\mathbb{N} = \{1;2;3;4;...\}\) and
\(\mathbb{N} \cup \{0\} = \{0;1;2;3;4;...\}\) have as many elements?

\begin{longtable}[]{@{}
  >{\raggedright\arraybackslash}p{(\columnwidth - 0\tabcolsep) * \real{0.06}}@{}}
\toprule
\endhead
Well, let \(\mathbb{N}\) be our hotel, and \(\mathbb{N} \cup \{0\}\) be
our set of visitors. That is, we imagine that an infinite amount of
people show up at Hotel Hilbert. Each of them is wearing a T-shirt with
a number written on it. The first visitor has a T-shirt with 0, the
second one has a T-shirt with 1, and so on\ldots{} Can we give everyone
a room in a way that makes the hotel complete? Well, we could start by
giving the room numbered \(n\) to the person with the T-shirt \(n\).
That is, the person with T-shirt 1 gets room 1, the person with T-shirt
33 gets room 33 and so on\ldots{} And then, on the door step, remains
one lonely visitor, the one who had the misfortune to wear T-shirt zero.
Seeing that, their friends decide to make a little room: the visitor in
room 1 moves to room 2, the visitor in room 2 moves to room 3, the
visitor in room 3 moves to room 4 and so on\ldots{} After they all move
following this pattern (which may take quite some time, but fortunately
a clever arrangement of black holes around the hotel allows for
appropriate space-time distorsion and makes this operation feasible),
every room of the hotel is occupied except for room number 1. Letting
our visitor number 0 occupy this room, we have a complete hotel and no
visitor left outside : we have a bijection between \(\{0;1;2;3;...\}\)
and \(\{1;2;3;...\}\). Here is another illustration of what is going on:
\includegraphics{https://functors.net/montessiel/wp-content/uploads/sites/11/2021/03/n1.png} \\ \addlinespace
Another way to look at what we just did is to say that if we take the
set \(\mathbb{N}\) and we add to it one element, 0, its size doesn't
change. We could express this fact by writing
\(\aleph_0 = \aleph_0 + 1\). (remember, \(\aleph_0\) is how we write the
infinite number representing the cardinal of
\(\mathbb{N}\)). \\ \addlinespace
We were trying to build an infinite cardinal number that is bigger that
\(\aleph_0\), and our first attempt has been unfruitful. Let's try
something else! We start, again, with an empty hotel. \\ \addlinespace
This time, imagine that two buses show up at the same time in front of
Hotel Hilbert. Each but has an infinite amount of passenger: one bus has
seats numbered 1, 2, 3,\ldots{} and so on to infinity and the second bus
has seats numbered -1,-2,-3,\ldots{} and so on\ldots{} Will all the
passengers be able to fit in the hotel?
\includegraphics{https://functors.net/montessiel/wp-content/uploads/sites/11/2021/03/two-buses.png}
Again, I encourage you to take some time and ponder this to yourself
before reading further. \\ \addlinespace
\bottomrule
\end{longtable}

The organizers of both trips (they are the passengers labeled -1 and 1)
deliber for a little while on the (infinite) parking lot of the hotel.
Fortunately, both buses are parked on small black holes, which helps
make the wait bearable for the passengers.

They come to a conclusion: passengers of the ``negative bus'' will
occupy even numbered rooms and passengers of the ``positive'' bus will
occupy odd numbered rooms. That is, passenger 1 will take room 1,
passenger -1 will take room 2, passenger 2 will take room 3, passenger
-2 will take room 4, and so on\ldots{}

Again, after they explain the process to their co-travelers, everyone
reachers their room peacefully and the hotel ends up complete. Here is
an other illustration of the strategy:\\
\includegraphics{https://functors.net/montessiel/wp-content/uploads/sites/11/2021/03/nn-2.png}\\
So this time, what we managed to fit two copies of the set
\(\mathbb{N}\) inside itself. We could translate this operation in terms
of cardinal numbers: \(\aleph_0 + \aleph_0 = \aleph_0\).

It may seem tempting to give up at this point. If even two copies of
\(\mathbb{N}\) can fit in \(\mathbb{N}\), can we really hope to find a
bigger infinity? Well, let's get a bit more creative.

\hypertarget{and-beyond}{%
\subsubsection{\ldots{} and beyond!}\label{and-beyond}}

We will now have to expand a bit more on our analogy:\\
One more time, we forget about our last thought experiment and consider
the hotel empty.\\
A simple good old infinite bus of visitors shows up at the hotel, and
each visitor is numbered as 1,2,3,\ldots{} and so on. But this time,
they're here for business. In fact, they are all spies and want to be
able to share information to exactly the people that they want.\\
In order to implement that, they ask that each possible combination of
them be attributed a specific room. For instance, if visitors number
1,54,256 and 10000003 want to talk, there is one room to which only they
have access. Of course, there is also a different room to which only
visitors 1,54 and 256 have access (they actually use it to prepare
visitor 10000003's surprise birthday party), and another room to which
only visitors 1, 54, 256 and 10000004 have access, and so on for any
possible set of visitor.\\
In fact, they even want meeting rooms for infinite sets of visitors. For
instance, odd numbered visitors sometimes like to gather without ``the
evens.''\\
Finally, for some classified, top secret reason, they also ask for a
room to which absolutely no visitor has access.

In other words, we need to find a bijection between \(\mathbb{N}\) and
the ``set of subsets of \(\mathbb{N}\),'' which we shall denote
\(P(\mathbb{N})\). Here is an illustration of the set of subsets of
\(\{1;2;3\}\), \(P(\{1;2;3\})\):\\
\includegraphics{https://functors.net/montessiel/wp-content/uploads/sites/11/2021/03/p3.png}

Each element of \(P(\{1;2;3\})\) is itself a set. Note that it also
includes an \emph{empty set} which contains nothing. We write
\(\emptyset\) to refer to it. You can also notice that the set
\(P(\{1;2;3\})\) has \(8 = 2^3\) elements. This \emph{might} be relevant
later on.

Back to Hotel Hilbert, we are dealing with the same construction, but
this time we are working with infinite sets. So our question follows: is
it possible to fulfill the Spy Workshop's request? As you will have
guessed, I encourage you to take a little while to ponder this by
yourself before you proceed further.

\begin{center}\rule{0.5\linewidth}{0.5pt}\end{center}

Well, this time, it's hard to think right away of an organisation method
that will solve our problem. We may use a different method: we will
start by assuming that actually have a way to organise our hotel into
this infinite conference center (that is, we will assume that we have a
bijection between \(\mathbb{N}\) and \(P(\mathbb{N})\)) and we will try
to draw conclusions from this assumption. A few things may then happen:
* We may gain a better understanding of what a solution to the problem
looks like, and that would give us hints to find such a solution. * We
may arrive to a contradiction, some sort of absurd statement. This would
prove that a solution to the problem doesn't exist. * We may learn
nothing of value and adopt another strategy to solve our problem.

So, we assume for now that each room has been attributed a set of
visitors. Remember that rooms of the hotel are all attributed one
specific natural number, and that the same goes for visitors. If \(n\)
is a natural number, we write \(R_n\) for the set of visitors that may
enter room number \(n\). For instance, we read on the picture below that
\(R_{1728}\) is the set of prime numbers. (note that this is just an
example that the number may very well have been a different one)
\includegraphics{https://functors.net/montessiel/wp-content/uploads/sites/11/2021/03/room1723-2.png}

But then, something happens. Visitor number 42 realizes that they are
not welcome in room number 42. That is, 42 is not one of the elements of
\(R_{42}\). They feel a bit jealous of their friend, visitor 44, because
room 44 is the room that welcomes visitors whose number has only one
digit repeating a certain number of time, so visitor number 44 is
welcome in room number 44 (and so are visitors 222, 5555555 and
7777777777777777777777).

Visitor number 42 would like to gather visitors who are not welcome in
the room bearing their name. They look for the room that welcomes all
the visitors who do not have access to the room that has the same number
as them. So visitor number 42 is welcome in that room, but visitor
number 44 isn't. Recalling our previous example, visitor number 1728 is
welcome in this room, because 1728 is \emph{not} a prime number. While
we do not know the number of this room, we know that it must have one.
So we will simply write it \(k\).\\
\includegraphics{https://functors.net/montessiel/wp-content/uploads/sites/11/2021/03/roomk-1.png}

As some visitors arrive in room number \(k\), one of them asks: ``Wait!
Will visitor number \(k\) be joining us?''\\
``Of course not!'' Another one replies, ``people who can enter the room
bearing their number are not welcome here.''\\
``But this is room number \(k\), so if visitor number \(k\) is not
welcome here, then by the rules of the room they \emph{should} be
welcome here\ldots{} But if they are welcome here then the rules say
that they shouldn't be?'' concludes visitor number 42, with a glimpse of
fear in their eyes.

As those last words are uttered, the ground starts shaking. Dust and
lumps of plaster are falling from the ceiling, and the loud, deep voice
of the occupant of the room welcoming elements of \(\emptyset\), coming
from no particular location, yells: ``I have detected a logical
incoherence in this universe. Please wait peacefully for imminent
rapture.''

The visitors, not having paid attention to the second sentence, panick.
Chaos ensues. As they see visitor 1728 running around, flailing their
arms, visitor 42 wakes up on a couch in the lobby. They had fallen
asleep, waiting for the administrator of the hotel to find a solution to
the matter at hand.

\begin{center}\rule{0.5\linewidth}{0.5pt}\end{center}

This little story shows that the administrator will never manage to
access the request at hand. If they did, it would create a logical
incoherence, and so it is impossible that they do. There are simply not
enough rooms to accomodate every possible gathering of visitors.

Mathematically, this means that we can not fit the set \(P(\mathbb{N})\)
inside the set \(\mathbb{N}\). Another way to put it, is that
\(\aleph_0 < 2^{\aleph_0}\). Indeed, remember how we saw that the
cardinal of \(P(\{1;2;3\})\) was \(8 = 2^3\) ? Well we keep this formula
as a notation even for infinite cardinals, and so the cardinal of the
set \(P(\mathbb{N})\) is written \(2^{\aleph_0}\), since the cardinal of
\(\mathbb{N}\) is \(\aleph_0\).

\hypertarget{conclusion}{%
\subsubsection{Conclusion}\label{conclusion}}

We have found one new infinite cardinal number, strictly larger than
\(\aleph_0\). We may be tempted to call it \(\aleph_1\), but doubts
remains:\\
\textbf{\emph{Is \(2^{\aleph_0}\) equal to \(\aleph_1\)? That is, is
\(2^{\aleph_0}\) the smallest cardinal number strictly larger than
\(\aleph_0\)?}}

This question, raised by Georg Cantor (who invented/discovered most, if
not all, of what was discussed in this post) in 1878. A surprising
answer was found by logicians Kurt Gödel and Paul Cohen in the
\(\text{XX}^{\text{th}}\) century: using the common theory of set, it is
impossible to prove or disprove that \(2^{\aleph_0} = \aleph_1\). In
fact, one may freely assume either that it is true or that it is false,
and it should not affect the coherence of their mathematics.

Explaining the methods they used to prove this goes way beyond the scope
of this blog post, and so we will stop here, content with our discovery
that not only infinite cardinal numbers do exist, but that there is more
than one. In fact, with a bit of imagination, the interested reader
could prove that there is \emph{an infinity of infinite cardinal
numbers}.

\end{document}
